\documentclass[12pt]{article}
\setlength{\textheight}{9.80in}
\setlength{\textwidth}{6.40in}
\setlength{\oddsidemargin}{0.0mm}
\setlength{\evensidemargin}{1.0mm}
\setlength{\topmargin}{-0.6in}
\setlength{\parindent}{0.2in}
\setlength{\parskip}{1.5ex}
\newtheorem{defn}{Definition}
\renewcommand{\baselinestretch}{1.2}

\begin{document}

\bibliographystyle{prsty}

\thispagestyle{empty}

\title{ The {\it DORS}: a Discrete Ordinates Radiation Simulator}

\author{C. Godsalve \\
email:{\tt seagods@hotmail.com} }

\maketitle

\section{After the Download}


Without going into any details, the {\it The-DORS} is primarily intended as a simulator
for ultra-violet, visible, and infrared radiation in the Earth's atmosphere. 
The radiative transfer calculations are {\it one dimensonal}, that is to say they assume a ``flat Earth" having
an atmosphere that has tranlational symmetry. The properties of the atmosphere vary only with height.

{\it The-DORS} consists 
of three separate stages, {\it Dors1}, {\it Dors2}, and {\it Dors3}. {\it Dors1} sets up the
details of the atmosphere, {\it Dors2} sets up the boundary condition at the ground, and
{\it Dors3} actually does the radiative transfer calculations based on the output of {\it Dors1} and {\it Dors2}.
All the details of what these codes do are contained in three separate documents: here we only 
tell what to do once you have downloaded a copy of the {\it The-DORS}. I have written this document
 bearing in mind that the {\it The-DORS} may be used by undergraduate students or first
year postgraduate students that don't yet have a lot of experience in computing. Because of this, I have 
spelled out a lot of very basic stuff.

If you have not obtained this document by downloading the {\it The-DORS}, you may obtain the code, data, and documentation
by typing
\begin{verbatim}
git clone git@github.com:seagods/The-DORS.git[ENTER]
\end{verbatim} 
at a terminal. This requires a program called 
{\it Git-Fast version control system},
 which may be obtained from ``http://git-scm.com".

I shall suppose you have downloaded the {\it The-DORS} from {\it GitHub}, and you use the default directory name
of ``The-DORS". This contains the source code of the computer programs that
I have written: these need to be compiled together with additional programmes.
It also contains various data directories.

 So, the {\it DORS} has some dependencies, and this document is here to tell you
where to get the things that the {\it DORS} depends on. With these, you can get the thing up and running.
The {\it DORS} was developed to run on a PC running Linux, so getting the programs compiled
will be just as easy on a UNIX system. Many of  dependencies are  various bits of freely
available mathematical software which we shall give details of below. Apart from mathematical
software, the {\it DORS}
 depends on the {\it HITRAN}
molecular database \cite{RothmanETAL:Mybib}.

 In order to get started, there is a only a moderate
amount of work to do on a Linux or UNIX system. Whatever the operating system, you will need a C++ compiler.
The {\it Gnu Compiler Collection} (gcc) includes C++, and is  completely free. It requires a {\it POSIX} compliant system.
Linux and Mac OSX are { \it POSIX compliant}, and most Linux distributions come with gcc. 

 So what if you have
a PC running MS Windows? 
If this is the
case you won't automatically have a compiler available. (At the time of writing Microsoft visual C++ 2010 is
 downloadable for free.) Unfortunately, as it stands with the DORS, gcc is essential because of the need to
call FORTRAN functions from C++ via ``gfortran" which also comes with the gcc. To use gcc it you need 
to get over the {\it POSIX} stumbling block. 
You have one of two options: download and install {\it Cygwin}, or download a Linux ``flavor" or ``distro" 
 such as Fedora, OpenSuSe, Ubuntu, or Mandriva. There are many such ``flavors". It {\it should} be easy to make
your machine ``dual boot", so you can choose to run MS Windows or Linux when you restart.
If you  really don't want a dual boot machine you need to use {\it Cygwin} option. (Just use google or some other search engine
to find {\it Cygwin}.) At any rate, none of this should cost you a penny, and obviously {\it back up all
your data } if you go for dual boot! 


So if you are an MS Windows user  and you choose to install Linux, you should have the necessary compilers
with the distribution. If you
go for {\it Cygwin}, all
you need to do is (for free) download and install gcc.
If you search on gcc, you shall soon find it. Make sure that gfortran gets installed
along with g++. 
Either way, to make the transition to Linux, all you need to do is to pop into a library and borrow a basic book on Linux
 or UNIX, or just type ``UNIX", ``Linux", along with  ``tutorial" into a search engine.


Now, once you are in the command line (thats right --- no fancy
 windows style point and click) and in the {\it The-DORS"} directory type ``compile". You get a whole bunch of
error messages! Why? Well first there is that other mathematical software you need (its all free) 
and then there is the {\it HITRAN} molecular database. We shall list the
freely downloadable routines and where to get them at the end of this document,
but first we go on to {\it HITRAN}.

\section{What You Need from {\it HITRAN}}


One of the main features of  {\it The-DORS} is that it uses the 
{\it HITRAN} \cite{RothmanETAL:Mybib} molecular database to calculate molecular absorption of radiation.
Molecular absorption is important for nearly all Earth
atmosphere purposes, so {\it HITRAN} is needed. Your {\it The-DORS} directory has an empty directory
named {\it HITRAN}.
 Again, just use a search engine, and the {\it HITRAN} home page
should be there. You will need to fill in an online request form, and it may take a while for
you to receive an email telling you how to download it. It it is quite likely that you might want to 
download {\it HITRAN} as a stand alone data base. If so, download all the necessary files
and documentation and keep it entirely separate from {\it The-DORS}, then copy what is
necessary to your {\it The-DORS/HITRAN"} directory.

It may be that you will be given html access to {\it HITRAN} which will make life easy: however, just in case you need
to use {\it ftp} via the command line, I shall include some of the details  of what you need to know 
about using {\it ftp}.
I shall suppose you use have {\it ftp} available. 
You have received
 the {\it ftp} address via email and have typed {\it ftp whatever.edu} 
at the login prompt. 
For the user name, you just type ``anonymous", and {\it please} use your {\it real}
 email address as the password. If you haven't used ftp before, just type ``help" at the ftp prompt.
This gives you a list of {\it ftp} commands. 
You shall see ``cd", ``dir", and ``mget" in the list, just do ``help mget" for instance, and you are told
what ``mget" does.  If you normaly use MS Windows, note 
that {\it everything} in ftp, Linux,  and Cygwin is case sensitive. 
For instance, ftp will recognise 	``cd DirName" for ``change directory to
 DirName", but
will give you an error if you type ``CD DirName" or ``cd DIRNAME" instead. 

Change directory to pub/HITRAN2008. If you type ``ls" you get a list of what's in the
current directory.  You will see things on the left looking like ``-rw-r--r--r" to the left of each list
item. If that starts with the letter `d', then it's a directory. Change directory to
``Global\_Data".
The ``pwd" command tells you what directory you are in at any time.
 For directory names and files that have spaces, you need to put a double quote at the start and end of
the directory name. 

In your {\it The-DORS} you have directory called ``HITRAN" containing only a README file.
 This directory is for the data, {\it don't} put 
what anything you might have  just downloaded here. You will encounter  some {\it pdf} files that
 give you a list of references to quote
for any publications you make using {\it HITRAN}. Naturally, use these for any publications you make
using the output of the {\it DORS}. Now, you might want to download absolutely everything, after all you get
things like {\it JavaHAWKS} and copies of papers describing {\it HITRAN}. I shall only mention what you need
to download to get the {\it DORS} running. Go back to ``Global\_Data" and download ``molparam.txt".
This goes in your empty ``HITRAN" directory.  Now go back to ``/pub/HITRAN2008". This has a subdirectory,
also called ``HITRAN2008", go there, and then to ``By-Molecule", download the pdf, and then
go to ``Compressed-files". Download everything here, put them in your ``HITRAN" directory and
unzip them. You now have a lot of files with names like ``01-hit08.par". That is the archive file
for molecule 1 in {\it HITRAN 2008}. Two files have an extra ``\_f53"
in the names. Just edit this out of the file names.

Now, going back to {\it ftp}, there are some FORTRAN programs to download. These compute the
partition functions needed to calculate absorption lines.
We need to get to "HITRAN2008/Global\_Data/Fortran Programs for Partition Sums".
Download all these and leave them in your ``The-DORS" directory: don't put them in your {\it HITRAN}
sub-directory.
 If your operating 
system is case dependent as in UNIX, you need to change some of the names in order to compile it. That is
\begin{verbatim}
mv BD_ISO_2002.for BD_ISO_2002.FOR
mv BD_ISO_82_to_85.for BD_ISO_82_85.FOR
mv ISOTOPS_2002.CMN Isotops_2002.cmn
mv MOLEC.CMN Molec.cmn
mv SPECIES_2002.CMN Species_2002.cmn
gfortran -o TIPS TIPS_2009.for
gfortran -c TIPS_2009.for
\end{verbatim}
The example driver program is called {\it TIPS\_2009}, and you now have a stand alone executable file called {\it TIPS} which you can test. The rest of the 
programs files will go into the {\it Dors library}.

\section{Water Vapour Continuum}

You shall also need to go to http://rtweb.aer.com/ and download the $MT CKD$ water vapour continuum program (version 2.5). The continuum model has been under development since the original paper by Clough, Kneizys, and Davies \cite{MTCKD:Mybib}. The release notes provide plenty of references as to how the subject has evolved. I have modified the driver so it can be called
as a function from the main program. I find the code difficult to follow --- it is meant to be a subroutine for $LBLRTM$
which you can also obtaing from this site. Once you have downloaded it, overwrite the "cntnm_prog.f"  with the one I provide.
You then need to modify ``makefile.common". You need to change "-Wall" in ``FCFLAGS"  to "-fPIC -g -c -Wall" so it can go into the Dors library. Then run gmake as in the ``README" in the build directory. You will now have an object file in the ``.obj"
 sub-directory which you can copy to the ``dorslib" directory.





\section{Library Programs - Building the Dors Library}

As well as these {\it FORTRAN} programs, ``The-Dors" requires various mathematical programs which we shall
 combine into a library. The good news is
that they are all free, copyright free, and patent free. You can distribute them freely (as long as you don't say you wrote
 them or stupid stuff like that). If you do distribute them - say where they are from - and please don't edit them without saying so!  

However, I have chosen {\it not} to distribute
them with ``The-DORS" (The reason is I don't have the space in my {\it github} account to distribute
them from there.  Instead, ``The-DORS" file contains a subdirectory called  {\it dorslib}.
This contains a plain text file called {\it Instructions.txt}. It tells you what to do if you put your {\it The-Dors}
directory in your home directory. You will have to make your own adjustments if you put it elsewhere.
I have distributed {\it HUMLIK.f} \cite{BobWells:MyBib} which calculates Voigt profiles
 \footnote{See http://www.atm.ox.ac.uk/wells/voigt.html}

The Instructions.txt file tells you all the maths library routines you need to download, and where to download them from. 
(I may put them all on {\it zSHARE} so you can download them all as a single gzipped tarfile as they take up
too much space for to put them on a free github account).
As well as this
there are two scripts that will build them all into a single library archive by running {\it script1} and then {\it script2}
and finishing off the instructions in the {\it Instructions.txt} file.
Note that this library will contain the {\it HITRAN} subroutines and functions that you downloaded, and is linked
 with {\it -ldors} at the compilation stage. Note that some of the FORTRAN programs that you get from {\it HITRAN}
are actually incorporated into the main one (BD\_TIPS\_2003) via include statements. They are not compiled separately.

 In your {\it The-DORS} directory there 
is a script {\it compile} that compiles ``The-DORS". (Make sure your compile script is executable by doing ``chmod u+x compile"
at the command line.) Do likewise for the scripts in your {\it dorslib} director.






\bibliography{/home/daddio/Articles/Mybib}
\end{document}

